%%%%%%%%%%%%%%%%%%%%%%%%%%%%%%%%%%%%%%%%%%%%%%%%%%%%%%%%%%%%%%%%%%%%%%%%%%%
%                                                                         %
%           TEMPLATE LATEX PER TESI                                       %
%           ______________                                                %
%                                                                         %
%           Ultima revisione: 24 giugno 2019                              %
%           Revisori: G.Presti; L.A.Ludovico; F. Avanzini                 %
%                                                                         %
%%%%%%%%%%%%%%%%%%%%%%%%%%%%%%%%%%%%%%%%%%%%%%%%%%%%%%%%%%%%%%%%%%%%%%%%%%%

\documentclass[12pt,italian]{report}
\usepackage{tesi}

% CORSO DI LAUREA:
\def\myCDL{Corso di Laurea triennale in Informatica}

% TITOLO TESI:
\def\myTitle{Pattern di assegnamento ottimizzati per l'efficienza energetica in edge computing}

% AUTORE:
\def\myName{Manuel Parati}
\def\myMat{Matr. 958584}

% RELATORE E CORRELATORE:
\def\myRefereeA{Prof. Alberto Ceselli}
\def\myRefereeB{Prof. Marco Premoli}

% ANNO ACCADEMICO
\def\myYY{2021-2022}

% Il seguente comando introduce un elenco delle figure dopo l'indice
%\figurespagetrue

% Il seguente comando introduce un elenco delle tabelle dopo l'indice
%\tablespagetrue

%
%       PREAMBOLO
%       Inserire qui eventuali package da includere o definizioni di comandi personalizzati
%

% Package di formato
\usepackage[a4paper]{geometry}      % Formato del foglio
\usepackage[italian]{babel}         % Supporto per l'italiano
\usepackage[utf8]{inputenc}         % Supporto per UTF-8
%\usepackage[a-1b]{pdfx}            % File conforme allo standard PDF-A (obbligatorio per la consegna)

% Package per la grafica
\usepackage{graphicx}               % Funzioni avanzate per le immagini
\usepackage{hologo}                 % Bibtex logo with \hologo{BibTeX}
\usepackage{epsfig}                % Permette immagini in EPS
%\usepackage{xcolor}                % Gestione avanzata dei colori

% Package tipografici
\usepackage{amssymb,amsmath,amsthm} % Simboli matematici
\usepackage{listings}               % Scrittura di codice

% Package ipertesto
\usepackage{url}                    % Visualizza e rendere interattii gli URL
\usepackage{hyperref}               % Rende interattivi i collegamenti interni


\begin{document}

% Creazione automatica del frontespizio
\frontespizio
\beforepreface

%
%       RINGRAZIAMENTI
%

\prefacesection{Ringraziamenti}
Questa sezione, contiene i ringraziamenti.

%
%       Creazione automatica dell'indice
%

\afterpreface

% 
%       CAPITOLI
% 

\chapter{Introduzione}
\label{cap:introduzione}

Negli ultimi anni le reti di comunicazione hanno subito una rapida evoluzione, abbracciando la virtualizzazione dei sistemi come metodologia per ottimizzare l'efficienza delle risorse fisiche e le spese necessarie per gestire la rete, oltre che aumentare la qualità dell'esperienza per l'utente finale. Un'area di ricerca particolarmente promettente è quella riguardante il Mobile Edge Computing (MEC), che propone una versione distribuita del classico cloud computing, in cui i server applicativi sono istallati nell'edge della rete tramite un sistema di virtualizzazione, in modo da diminuire la distanza di rete tra gli utenti e i server, ed offrire quindi maggiore affidabilità e migliori prestazioni alle connessioni di rete, oltre che a ridurre la quantità di energia necessaria a gestire l'intera infrastruttura. Questo tipo di architettura migliora il servizio offerto all'utente riducendo drasticamente la latenza di connessione ai server e virtualizzando le operazioni di calcolo sul nodo più vicino, permettendo di avere un basso impatto energetico sui dispositivi mobili e quindi poter eseguire applicazioni computazionalmente pesanti utilizzando device a basso consumo. Diversi scenari applicativi traggono vantaggio dall'utilizzo dell'infrastruttura MEC, come la realtà aumentata o i veicoli a guida autonoma, o più in generale tutti quelli in cui sia necessaria elevata potenza di computazione e bassa latenza.


%
%   OBBIETTIVI
%
\section{Obbiettivi}
\label{sec:obbiettivi}

Si consideri una rete MEC formata da cluster di virtualizzazione dotati di capacità limitata ed un insieme di acces point (AP) sottoposti a traffico variabile nel tempo. Questo lavoro propone un modello di orchestrazione che assegni dinamicamente il traffico di ogni AP ad un nodo della rete, con l'obbiettivo di fornire un'alta qualità del servizio pur cercando di contenere i costi relativi alla gestione e minimizzando la quantità di energia utilizzata. Tale modello viene implementato utilizzando due diversi approcci euristici, che verranno approfondidi, analizzati e confrontati nel corso dei capitoli.


%
%   ORGANIZZAZIONE ELABORATO
%
\section{Struttura dell'elaborato}
\label{sec:organizzazione-elaborato}

Il documento è diviso in sei capitoli:
\begin{itemize}
    \item[\textbf{1.}] \textbf{\nameref{cap:introduzione}}: introduzione al lavoro svolto;
    \item[\textbf{2.}] \textbf{\nameref{cap:modellazione-sistema}}: fornisce una panoramica riguardo l'infrastruttura Mobile Edge Computing e illustra il funzionamento del modello di orchestrazione proposto;
    \item[\textbf{3.}] \textbf{\nameref{cap:modelli-ottimizzazione}}: definisce i modelli di ottimizzazione utilizzati dagli algoritmi euristici, mostrando la loro rappresentazione matematica e descrivendo nel dettaglio le variabili, i vincoli e l'obbiettivo da raggiungere;
    \item[\textbf{4.}] \textbf{\nameref{cap:algoritmi-euristici}}: illustra i due algoritmi euristici proposti, mostrando il loro pseudocodice e descrivendo le operazioni che svolgono;
    \item[\textbf{5.}] \textbf{\nameref{cap:analisi-sperimentale}}: inizialmente presenta il setup e i dati utilizzati, successivamente descrive i test e mostra i risultati ottenuti;
    \item[\textbf{6.}] \textbf{\nameref{cap:conclusioni}}: descrive dettagliatamente le conclusioni più rilevanti ottenute nella parte precedente.
\end{itemize}
Dopo l'ultimo capitolo, è presente l'appendice \textbf{\nameref{cap:modelli-di-ottimizzazione-framm}}, in cui sono presentati due modelli di ottimizzazione che propongono due diverse metodologie per frammentare un array in blocchi.

\chapter{Modellazione del sistema}
\label{cap:modellazione-sistema}

Questo capitolo fornisce una panoramica riguardo all'infrastruttura Mobile Edge Computing ed illustra il modello di orchestrazione proposto in questo lavoro.


%
%   INFRASTRUTTURA MEC
%
\section{L'infrastruttura Mobile Edge Computing}
\label{sec:infrastruttura-mec}

All'interno di un'infrastruttura Mobile Edge Computing (MEC) si trovano i cluster di virtualizzazione, spesso noti come `MEC facility' o più semplicemente `facility', e gli access point (AP). Le facility sono il luogo in cui avviene la virtualizzazione e sono composte dalle macchine virtuali (VM) su cui si eseguono le applicazioni degli utenti finali, mentre gli AP sono i dispositivi (per esempio le antenne wireless) a cui gli end point si collegano per poter ricevere il servizio. Ogni AP è associato ad una facility, a cui inoltra tutto il traffico che riceve: questo significa che ogni facility avrà in esecuzione nelle proprie VM tutte le applicazioni utilizzate dagli utenti connessi agli AP a lui associati. Ogni facility, per poter gestire il traffico che le viene inoltrato, deve utilizzare una quantità di energia direttamente proporzionale a tale domanda. Per questo motivo, ognuna di esse possiede dei pannelli fotovoltaici che generano una quantità di energia variabile nel tempo, dipendente dal numero di pannelli installati e dall'irraggiamento a cui sono sottoposti. L'energia prodotta può essere direttamente utilizzata oppure immagazzinata all'interno di alcune batterie, dotate di capacità limitata. Nel caso in cui l'energia disponibile, data dalla somma tra quella accumulata e quella generata, non basti a soddisfare la domanda, è possibile acquistarne altra ad un prezzo dipendente dalla facility e dall'istante temporale.

Data la natura mobile degli end point, che sono per esempio smartphone o laptop, il traffico a cui sono sottoposti gli AP varia nel tempo e di conseguenza cambia la domanda rivolta alle facility. Per questo motivo l'assegnamento viene effettuato dinamicamente, e lo strumento incaricato di svolgere tale compito è l'orchestratore, che implementa la logica definita dal modello di orchestrazione. Le azioni che svolge vengono chiamate `orchestrazioni' o `switch' e consistono nell'assegnare un AP ad una facility diversa da quella attuale, provocando il ridimensionamento della potenza delle VM in termini del numero di processori e memoria disponibile, e la migrazione del loro stato. Il ridimensionamento è dovuto alla variazione di domanda da gestire, mentre la migrazione dello stato è necessaria per avere in esecuzione in ogni facility le applicazioni utilizzate dagli utenti connessi agli AP che le sono assegnati. Il costo prodotto dalla migrazione prende il nome di `costo di migrazione'.

\begin{figure}[t]
    \centering
    \includegraphics[width = 150mm]{img/esempio-infrastruttura-mec.jpg}
    \caption{Esempio di un'infrastruttura Mobile Edge Computing (MEC).}
    \label{fig:architettura-mec}
\end{figure}

Nella \autoref{fig:architettura-mec} è presente un'infrastruttura MEC composta da due facility e tre AP, a cui sono collegati diversi dispositivi mobili di vario genere. Si noti come ogni dispositivo invii il proprio traffico all'AP a cui è collegato, e come tutto il traffico ricevuto da ogni AP sia inoltrato alla stessa facility. Si può osservare, per esempio, come al secondo AP siano connessi un laptop e due smartphone, mentre al terzo un'automobile smart ed un tablet. Nella figura i due AP sono assegnati alla stessa facility (la seconda) che riceverà e dovrà quindi gestire il traffico di tutti e cinque i dispositivi. Si noti infine come sia presente un cloud centralizzato che gestisce e sincronizza le varie facility.


%
%   INFRASTRUTTURA MEC
%
\section{Modello di orchestrazione proposto}
\label{sec:modello-di-orchestrazione-proposto}

Il modello di orchestrazione proposto rappresenta una variante di quello presentato negli articoli \cite{assignment-patterns} e \cite{analytics-mec}, in cui nell'effettuare le scelte di orchestrazione si tiene in considerazione anche l'utilizzo dell'energia da parte delle facility.

Come illustrato in \cite{analytics-mec}, all'interno del modello viene introdotta una discretizzazione temporale che permette di effettuare migrazioni solo in determinati istanti, per esempio ogni 15 minuti. Questa scelta è ragionevole, dato che in caso contrario si pagherebbe un costo troppo elevato in termini di migrazione e spese generali dovute alla gestione del traffico. L'orizzonte temporale viene quindi rappresentato da un insieme di time-slot della stessa durata.

L'obbiettivo del modello è quello di effettuare assegnamenti che permettano di raggiungere un buon compromesso tra: ottenere una buona qualità del servizio (cioè bassa latenza per l'utente finale), costi di migrazione contenuti e buona gestione dell'energia da parte delle facility. Nell'effettuare le scelte bisogna tenere in considerazione che la domanda degli AP e l'energia prodotta dalle facility varia nel corso del tempo, e che ogni facility possiede un limite massimo di domanda che è in grado di soddisfare simultaneamente e una quantità di energia limite che può mantenere nello stesso momento all'interno delle proprie batterie. Si supponga inoltre che le facility, per gestire una unità di traffico, debbano utilizzare una unità di energia.

\begin{figure}[t]
    \centering
    \includegraphics[width = 150mm]{img/esempio-assegnamenti.jpg}
    \caption{Esempio funzionamento del modello di orchestrazione.}
    \label{fig:esempio-assegnamenti}
\end{figure}


%
%   ESEMPIO FUNZIONAMENTO
%
\subsection{Esempio del funzionamento}
\label{sub-sec:esempio-funzionamento}

Una semplice applicazione di esempio è presentata nella \autoref{fig:esempio-assegnamenti}. Come si può osservare, è presente una rete MEC con due facility (\texttt{K1} e \texttt{K2}) e quattro AP (\texttt{A}, \texttt{B}, \texttt{C}, \texttt{D}) considerata in due time-slot consecutivi (\texttt{t}=0, 1). Gli end point connessi agli AP sono rappresentati con dei piccoli cerchi, e si supponga che inizialmente le batterie siano vuote in entrambe le facility. Nel primo time-slot (\texttt{t}=1) gli AP \texttt{A} e \texttt{B} sono associati alla facility \texttt{K1} mentre \texttt{C} e \texttt{D} a \texttt{K2}, quindi tutti gli utenti che sono connessi ad \texttt{A} e \texttt{B} avranno le proprie applicazioni in esecuzione su una VM presente in \texttt{K1}, mentre quelli connessi a \texttt{C} e \texttt{D} le avranno in \texttt{K2}. Questi assegnamenti permettono di avere una buona latenza dato che ogni AP è connesso alla facility più vicina, ma anche una buona gestione energetica, in quanto nella facility \texttt{K1} vengono prodotte 10 unità di energia ed utilizzate solo 7 mentre in \texttt{K2} se ne producono 11 e utilizzano 9. In entrambi i casi avanzano delle unità (3 in \texttt{K1} e 2 in \texttt{K2}) che vengono immagazzinate nelle batterie per poter essere spese nei prossimi time-slot. Successivamente tre utenti connessi all'AP \texttt{B} si spostano e si agganciano a \texttt{C}: a questo punto \texttt{K2} riceve una richiesta di domanda eccessiva che non riesce a gestire e di conseguenza è necessario effettuare un'orchestrazione per ribilanciare il traffico. Questo comporta l'assegnare \texttt{C} a \texttt{K1}, ridimensionare le VM (aumentarne la potenza in \texttt{K1} e diminuirla in \texttt{K2}) e sincronizzare le facility trasferendo lo stato delle VM riguardanti tutti gli utenti connessi a \texttt{C} da \texttt{K2} a \texttt{K1}. In un contesto reale, il modello deve prevenire una situazione di questo tipo effettuando orchestrazioni che tengano in considerazione la futura domanda dei vari AP, dato che questa circostanza va ad inficiare la qualità complessiva del servizio. Nel secondo time-slot (\texttt{t}=1) si avranno quindi gli AP \texttt{A}, \texttt{B} e \texttt{C} assegnati alla facility \texttt{K1}, mentre \texttt{D} a \texttt{K2}. Dal punto di vista energetico, la facility \texttt{K1} riesce a soddisfare la domanda di traffico (9 unità) utilizzando l'energia presente nella batteria e quella prodotta in questo time-slot (8 + 3 unità complessive). Per quanto riguarda invece \texttt{K2}, l'energia disponibile (3 + 2 unità) non è sufficiente e quindi è necessario acquistare una ulterione unità. Da notare come l'energia potrebbe avere costi diversi nel corso dei time-slot, e di conseguenza potrebbe risultare conveniente acquistarla quando il costo è basso per immagazzinarla ed utilizzarla successivamente.


%
%   IMPLEMENTAZIONE
%
\subsection{Implementazione}
\label{sub-sec:implementazione}

Il comportamento del modello di orchestrazione fin qui descritto viene implementato attraverso un modello di programmazione lineare chiamato `modello di assegnamento dinamico' e descritto dettagliatamente nella \autoref{sec:modello-completo}. Al modello viene fornita come input un'instanza del problema (formata dati dati riguardanti l'infrastruttura MEC presa in considerazione e, per ogni time-slot, il traffico rivolto agli AP e la produzione di energia di ogni facility) e ritorna in output il piano degli assegnamenti ottimale. Questa rappresenta la miglior metodologia per risolvere il problema in quanto garantisce di ottenere la soluzione ottima, ma a causa della sua natura combinatoria potrebbe essere inutilizzabile in deterimnate situazioni. Infatti quando il numero di time-slot presi in considerazione oppure la quantità di AP o facility aumenta, il tempo di calcolo necessario ad arrivare alla soluzione ottima cresce esponenzialmente fino a diventare proibitivo. Per questo motivo vengono successivamente proposte due euristiche, che semplificano il problema con l'obbiettivo di abbassare la complessità del modello ed ottenere una soluzione non troppo peggiore rispetto a quella ottima. Nella \autoref{sec:semplificazione-modello} viene presentata la prima idea euristica, che mira ad abbassare la complessità semplificando il modello utilizzato, mentre nella \autoref{sec:aggregamento-time-slot} la seconda, che aggrega i time-slot dell'istanza per ottenerne una più facilmente risolvibile.


%
%       BIBLIOGRAFIA
%

\bibliographystyle{unsrt}
\bibliography{bibliografia}
\addcontentsline{toc}{chapter}{Bibliografia}


% Pagina dichiusura del LIM
% \closingpage

\end{document}
