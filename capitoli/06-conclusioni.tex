\chapter{Conclusioni}
\label{cap:conclusioni}

In conclusione, questo lavoro di tesi ha permesso di mostrare come sia possibile estendere in modo compatto i modelli di ottimizzazione presenti in letteratura, al fine di gestire esplicitamente la quantità di energia utilizzata nei siti MEC e considerando uno scenario di produzione e accumulo energetico. La fase sperimentale ha inoltre mostrato come le due euristiche proposte rappresentino valide implementazioni per il modello di orchestrazione considerato, in quanto permettono di abbassare notevolmente il tempo di calcolo e peggiorare le soluzioni solo di pochi punti percentuali. Negli esperimenti condotti, è stato anche sottolineato come la loro configurazione, vale a dire al modo in cui viene suddivisa l'istanza nella prima euristica e in che modo vengono aggregate le fascie temporali nella seconda, impatti notevolmente sui risultati ed i tempi di calcolo ottenuti, e mostra come nessuno dei due algoritmi abbia un impatto predominante sull'altro.
