%%%%%%%%%%%%%%%%%%%%%%%%%%%%%%%%%%%%%%%%%%%%%%%%%%%%%%%%%%%%%%%%%%%%%%%%%%%
%                                                                         %
%           TEMPLATE LATEX PER TESI                                       %
%           ______________                                                %
%                                                                         %
%           Ultima revisione: 24 giugno 2019                              %
%           Revisori: G.Presti; L.A.Ludovico; F. Avanzini                 %
%                                                                         %
%%%%%%%%%%%%%%%%%%%%%%%%%%%%%%%%%%%%%%%%%%%%%%%%%%%%%%%%%%%%%%%%%%%%%%%%%%%

\documentclass[12pt,italian]{report}
\usepackage{tesi}

% CORSO DI LAUREA:
\def\myCDL{Corso di Laurea triennale in Informatica}

% TITOLO TESI:
\def\myTitle{Pattern di assegnamento ottimizzati per l'efficienza energetica in edge computing}

% AUTORE:
\def\myName{Manuel Parati}
\def\myMat{Matr. 958584}

% RELATORE E CORRELATORE:
\def\myRefereeA{Prof. Alberto Ceselli}
\def\myRefereeB{Prof. Marco Premoli}

% ANNO ACCADEMICO
\def\myYY{2021-2022}

% Il seguente comando introduce un elenco delle figure dopo l'indice
%\figurespagetrue

% Il seguente comando introduce un elenco delle tabelle dopo l'indice
%\tablespagetrue

%
%       PREAMBOLO
%       Inserire qui eventuali package da includere o definizioni di comandi personalizzati
%

% Package di formato
\usepackage[a4paper]{geometry}      % Formato del foglio
\usepackage[italian]{babel}         % Supporto per l'italiano
\usepackage[utf8]{inputenc}         % Supporto per UTF-8
%\usepackage[a-1b]{pdfx}            % File conforme allo standard PDF-A (obbligatorio per la consegna)

% Package per la grafica
\usepackage{graphicx}               % Funzioni avanzate per le immagini
\usepackage{hologo}                 % Bibtex logo with \hologo{BibTeX}
\usepackage{epsfig}                % Permette immagini in EPS
%\usepackage{xcolor}                % Gestione avanzata dei colori

% Package tipografici
\usepackage{amssymb,amsmath,amsthm} % Simboli matematici
\usepackage{listings}               % Scrittura di codice

% Package ipertesto
\usepackage{url}                    % Visualizza e rendere interattii gli URL
\usepackage{hyperref}               % Rende interattivi i collegamenti interni

% Package algoritmi
\usepackage{algorithm,algpseudocode}

% Package appendice
\usepackage[toc,page]{appendix}

\numberwithin{equation}{section}    % Numerazione delle equazione indicando la sezione
\allowdisplaybreaks                 % Permette di spezzare i modelli quanto necessario

% Permette di avere "Input" e "Output" negli algoritmi invece di "Require" e "Ensure"
\renewcommand{\algorithmicrequire}{\textbf{Input:}}
\renewcommand{\algorithmicensure}{\textbf{Output:}}

% Definisce il testo da inserire nei riferimenti agli algoritmi
\newcommand{\algorithmautorefname}{algoritmo}

% Permette di avere la scritta "Appendici" invece di "Appendices".
\renewcommand{\appendixpagename}{Appendici}
\renewcommand{\appendixtocname}{Appendici}

% Permette di avere "Algoritmo" nella caption.
\floatname{algorithm}{Algoritmo}

% Permette di fare riferimenti alle righe degl algoritmi.
\makeatletter
\patchcmd{\ALG@step}{\addtocounter{ALG@line}{1}}{\refstepcounter{ALG@line}}{}{}
\newcommand{\ALG@lineautorefname}{linea}
\makeatother


\begin{document}

% Creazione automatica del frontespizio
\frontespizio
\beforepreface

%
%       RINGRAZIAMENTI
%

\prefacesection{Ringraziamenti}
Questa sezione, contiene i ringraziamenti.

%
%       Creazione automatica dell'indice
%

\afterpreface

% 
%       CAPITOLI
% 

\chapter{Introduzione}
\label{cap:introduzione}

Negli ultimi anni le reti di comunicazione hanno subito una rapida evoluzione, abbracciando la virtualizzazione dei sistemi come metodologia per ottimizzare l'efficienza delle risorse fisiche e le spese necessarie per gestire la rete, oltre che aumentare la qualità dell'esperienza per l'utente finale. Un'area di ricerca particolarmente promettente è quella riguardante il Mobile Edge Computing (MEC), che propone una versione distribuita del classico cloud computing, in cui i server applicativi sono istallati nell'edge della rete tramite un sistema di virtualizzazione, in modo da diminuire la distanza di rete tra gli utenti e i server, ed offrire quindi maggiore affidabilità e migliori prestazioni alle connessioni di rete, oltre che a ridurre la quantità di energia necessaria a gestire l'intera infrastruttura. Questo tipo di architettura migliora il servizio offerto all'utente riducendo drasticamente la latenza di connessione ai server e virtualizzando le operazioni di calcolo sul nodo più vicino, permettendo di avere un basso impatto energetico sui dispositivi mobili e quindi poter eseguire applicazioni computazionalmente pesanti utilizzando device a basso consumo. Diversi scenari applicativi traggono vantaggio dall'utilizzo dell'infrastruttura MEC, come la realtà aumentata o i veicoli a guida autonoma, o più in generale tutti quelli in cui sia necessaria elevata potenza di computazione e bassa latenza.


%
%   OBBIETTIVI
%
\section{Obbiettivi}
\label{sec:obbiettivi}

Si consideri una rete MEC formata da cluster di virtualizzazione dotati di capacità limitata ed un insieme di acces point (AP) sottoposti a traffico variabile nel tempo. Questo lavoro propone un modello di orchestrazione che assegni dinamicamente il traffico di ogni AP ad un nodo della rete, con l'obbiettivo di fornire un'alta qualità del servizio pur cercando di contenere i costi relativi alla gestione e minimizzando la quantità di energia utilizzata. Tale modello viene implementato utilizzando due diversi approcci euristici, che verranno approfondidi, analizzati e confrontati nel corso dei capitoli.


%
%   ORGANIZZAZIONE ELABORATO
%
\section{Struttura dell'elaborato}
\label{sec:organizzazione-elaborato}

Il documento è diviso in sei capitoli:
\begin{itemize}
    \item[\textbf{1.}] \textbf{\nameref{cap:introduzione}}: introduzione al lavoro svolto;
    \item[\textbf{2.}] \textbf{\nameref{cap:modellazione-sistema}}: fornisce una panoramica riguardo l'infrastruttura Mobile Edge Computing e illustra il funzionamento del modello di orchestrazione proposto;
    \item[\textbf{3.}] \textbf{\nameref{cap:modelli-ottimizzazione}}: definisce i modelli di ottimizzazione utilizzati dagli algoritmi euristici, mostrando la loro rappresentazione matematica e descrivendo nel dettaglio le variabili, i vincoli e l'obbiettivo da raggiungere;
    \item[\textbf{4.}] \textbf{\nameref{cap:algoritmi-euristici}}: illustra i due algoritmi euristici proposti, mostrando il loro pseudocodice e descrivendo le operazioni che svolgono;
    \item[\textbf{5.}] \textbf{\nameref{cap:analisi-sperimentale}}: inizialmente presenta il setup e i dati utilizzati, successivamente descrive i test e mostra i risultati ottenuti;
    \item[\textbf{6.}] \textbf{\nameref{cap:conclusioni}}: descrive dettagliatamente le conclusioni più rilevanti ottenute nella parte precedente.
\end{itemize}
Dopo l'ultimo capitolo, è presente l'appendice \textbf{\nameref{cap:modelli-di-ottimizzazione-framm}}, in cui sono presentati due modelli di ottimizzazione che propongono due diverse metodologie per frammentare un array in blocchi.

\chapter{Modellazione del sistema}
\label{cap:modellazione-sistema}

Questo capitolo fornisce una panoramica riguardo all'infrastruttura Mobile Edge Computing ed illustra il modello di orchestrazione proposto in questo lavoro.


%
%   INFRASTRUTTURA MEC
%
\section{L'infrastruttura Mobile Edge Computing}
\label{sec:infrastruttura-mec}

All'interno di un'infrastruttura Mobile Edge Computing (MEC) si trovano i cluster di virtualizzazione, spesso noti come `MEC facility' o più semplicemente `facility', e gli access point (AP). Le facility sono il luogo in cui avviene la virtualizzazione e sono composte dalle macchine virtuali (VM) su cui si eseguono le applicazioni degli utenti finali, mentre gli AP sono i dispositivi (per esempio le antenne wireless) a cui gli end point si collegano per poter ricevere il servizio. Ogni AP è associato ad una facility, a cui inoltra tutto il traffico che riceve: questo significa che ogni facility avrà in esecuzione nelle proprie VM tutte le applicazioni utilizzate dagli utenti connessi agli AP a lui associati. Ogni facility, per poter gestire il traffico che le viene inoltrato, deve utilizzare una quantità di energia direttamente proporzionale a tale domanda. Per questo motivo, ognuna di esse possiede dei pannelli fotovoltaici che generano una quantità di energia variabile nel tempo, dipendente dal numero di pannelli installati e dall'irraggiamento a cui sono sottoposti. L'energia prodotta può essere direttamente utilizzata oppure immagazzinata all'interno di alcune batterie, dotate di capacità limitata. Nel caso in cui l'energia disponibile, data dalla somma tra quella accumulata e quella generata, non basti a soddisfare la domanda, è possibile acquistarne altra ad un prezzo dipendente dalla facility e dall'istante temporale.

Data la natura mobile degli end point, che sono per esempio smartphone o laptop, il traffico a cui sono sottoposti gli AP varia nel tempo e di conseguenza cambia la domanda rivolta alle facility. Per questo motivo l'assegnamento viene effettuato dinamicamente, e lo strumento incaricato di svolgere tale compito è l'orchestratore, che implementa la logica definita dal modello di orchestrazione. Le azioni che svolge vengono chiamate `orchestrazioni' o `switch' e consistono nell'assegnare un AP ad una facility diversa da quella attuale, provocando il ridimensionamento della potenza delle VM in termini del numero di processori e memoria disponibile, e la migrazione del loro stato. Il ridimensionamento è dovuto alla variazione di domanda da gestire, mentre la migrazione dello stato è necessaria per avere in esecuzione in ogni facility le applicazioni utilizzate dagli utenti connessi agli AP che le sono assegnati. Il costo prodotto dalla migrazione prende il nome di `costo di migrazione'.

\begin{figure}[t]
    \centering
    \includegraphics[width = 150mm]{img/esempio-infrastruttura-mec.jpg}
    \caption{Esempio di un'infrastruttura Mobile Edge Computing (MEC).}
    \label{fig:architettura-mec}
\end{figure}

Nella \autoref{fig:architettura-mec} è presente un'infrastruttura MEC composta da due facility e tre AP, a cui sono collegati diversi dispositivi mobili di vario genere. Si noti come ogni dispositivo invii il proprio traffico all'AP a cui è collegato, e come tutto il traffico ricevuto da ogni AP sia inoltrato alla stessa facility. Si può osservare, per esempio, come al secondo AP siano connessi un laptop e due smartphone, mentre al terzo un'automobile smart ed un tablet. Nella figura i due AP sono assegnati alla stessa facility (la seconda) che riceverà e dovrà quindi gestire il traffico di tutti e cinque i dispositivi. Si noti infine come sia presente un cloud centralizzato che gestisce e sincronizza le varie facility.


%
%   INFRASTRUTTURA MEC
%
\section{Modello di orchestrazione proposto}
\label{sec:modello-di-orchestrazione-proposto}

Il modello di orchestrazione proposto rappresenta una variante di quello presentato negli articoli \cite{assignment-patterns} e \cite{analytics-mec}, in cui nell'effettuare le scelte di orchestrazione si tiene in considerazione anche l'utilizzo dell'energia da parte delle facility.

Come illustrato in \cite{analytics-mec}, all'interno del modello viene introdotta una discretizzazione temporale che permette di effettuare migrazioni solo in determinati istanti, per esempio ogni 15 minuti. Questa scelta è ragionevole, dato che in caso contrario si pagherebbe un costo troppo elevato in termini di migrazione e spese generali dovute alla gestione del traffico. L'orizzonte temporale viene quindi rappresentato da un insieme di time-slot della stessa durata.

L'obbiettivo del modello è quello di effettuare assegnamenti che permettano di raggiungere un buon compromesso tra: ottenere una buona qualità del servizio (cioè bassa latenza per l'utente finale), costi di migrazione contenuti e buona gestione dell'energia da parte delle facility. Nell'effettuare le scelte bisogna tenere in considerazione che la domanda degli AP e l'energia prodotta dalle facility varia nel corso del tempo, e che ogni facility possiede un limite massimo di domanda che è in grado di soddisfare simultaneamente e una quantità di energia limite che può mantenere nello stesso momento all'interno delle proprie batterie. Si supponga inoltre che le facility, per gestire una unità di traffico, debbano utilizzare una unità di energia.

\begin{figure}[t]
    \centering
    \includegraphics[width = 150mm]{img/esempio-assegnamenti.jpg}
    \caption{Esempio funzionamento del modello di orchestrazione.}
    \label{fig:esempio-assegnamenti}
\end{figure}


%
%   ESEMPIO FUNZIONAMENTO
%
\subsection{Esempio del funzionamento}
\label{sub-sec:esempio-funzionamento}

Una semplice applicazione di esempio è presentata nella \autoref{fig:esempio-assegnamenti}. Come si può osservare, è presente una rete MEC con due facility (\texttt{K1} e \texttt{K2}) e quattro AP (\texttt{A}, \texttt{B}, \texttt{C}, \texttt{D}) considerata in due time-slot consecutivi (\texttt{t}=0, 1). Gli end point connessi agli AP sono rappresentati con dei piccoli cerchi, e si supponga che inizialmente le batterie siano vuote in entrambe le facility. Nel primo time-slot (\texttt{t}=1) gli AP \texttt{A} e \texttt{B} sono associati alla facility \texttt{K1} mentre \texttt{C} e \texttt{D} a \texttt{K2}, quindi tutti gli utenti che sono connessi ad \texttt{A} e \texttt{B} avranno le proprie applicazioni in esecuzione su una VM presente in \texttt{K1}, mentre quelli connessi a \texttt{C} e \texttt{D} le avranno in \texttt{K2}. Questi assegnamenti permettono di avere una buona latenza dato che ogni AP è connesso alla facility più vicina, ma anche una buona gestione energetica, in quanto nella facility \texttt{K1} vengono prodotte 10 unità di energia ed utilizzate solo 7 mentre in \texttt{K2} se ne producono 11 e utilizzano 9. In entrambi i casi avanzano delle unità (3 in \texttt{K1} e 2 in \texttt{K2}) che vengono immagazzinate nelle batterie per poter essere spese nei prossimi time-slot. Successivamente tre utenti connessi all'AP \texttt{B} si spostano e si agganciano a \texttt{C}: a questo punto \texttt{K2} riceve una richiesta di domanda eccessiva che non riesce a gestire e di conseguenza è necessario effettuare un'orchestrazione per ribilanciare il traffico. Questo comporta l'assegnare \texttt{C} a \texttt{K1}, ridimensionare le VM (aumentarne la potenza in \texttt{K1} e diminuirla in \texttt{K2}) e sincronizzare le facility trasferendo lo stato delle VM riguardanti tutti gli utenti connessi a \texttt{C} da \texttt{K2} a \texttt{K1}. In un contesto reale, il modello deve prevenire una situazione di questo tipo effettuando orchestrazioni che tengano in considerazione la futura domanda dei vari AP, dato che questa circostanza va ad inficiare la qualità complessiva del servizio. Nel secondo time-slot (\texttt{t}=1) si avranno quindi gli AP \texttt{A}, \texttt{B} e \texttt{C} assegnati alla facility \texttt{K1}, mentre \texttt{D} a \texttt{K2}. Dal punto di vista energetico, la facility \texttt{K1} riesce a soddisfare la domanda di traffico (9 unità) utilizzando l'energia presente nella batteria e quella prodotta in questo time-slot (8 + 3 unità complessive). Per quanto riguarda invece \texttt{K2}, l'energia disponibile (3 + 2 unità) non è sufficiente e quindi è necessario acquistare una ulterione unità. Da notare come l'energia potrebbe avere costi diversi nel corso dei time-slot, e di conseguenza potrebbe risultare conveniente acquistarla quando il costo è basso per immagazzinarla ed utilizzarla successivamente.


%
%   IMPLEMENTAZIONE
%
\subsection{Implementazione}
\label{sub-sec:implementazione}

Il comportamento del modello di orchestrazione fin qui descritto viene implementato attraverso un modello di programmazione lineare chiamato `modello di assegnamento dinamico' e descritto dettagliatamente nella \autoref{sec:modello-completo}. Al modello viene fornita come input un'instanza del problema (formata dati dati riguardanti l'infrastruttura MEC presa in considerazione e, per ogni time-slot, il traffico rivolto agli AP e la produzione di energia di ogni facility) e ritorna in output il piano degli assegnamenti ottimale. Questa rappresenta la miglior metodologia per risolvere il problema in quanto garantisce di ottenere la soluzione ottima, ma a causa della sua natura combinatoria potrebbe essere inutilizzabile in deterimnate situazioni. Infatti quando il numero di time-slot presi in considerazione oppure la quantità di AP o facility aumenta, il tempo di calcolo necessario ad arrivare alla soluzione ottima cresce esponenzialmente fino a diventare proibitivo. Per questo motivo vengono successivamente proposte due euristiche, che semplificano il problema con l'obbiettivo di abbassare la complessità del modello ed ottenere una soluzione non troppo peggiore rispetto a quella ottima. Nella \autoref{sec:semplificazione-modello} viene presentata la prima idea euristica, che mira ad abbassare la complessità semplificando il modello utilizzato, mentre nella \autoref{sec:aggregamento-time-slot} la seconda, che aggrega i time-slot dell'istanza per ottenerne una più facilmente risolvibile.

\chapter{Modelli di ottimizzazione}
\label{cap:modelli-ottimizzazione}

In questo capitolo vengono introdotti i modelli di ottimizzazione utilizzati.


%
%   MODELLO COMPLETO
%
\section{Modello completo}
\label{sec:modello-completo}

\dots

\begin{align}
    \min\quad       & \alpha \sum_{t \in T} \sum_{i \in A} \sum_{\substack{(j,k) \in \\ K \times K}}d^{t}_{i} l_{jk} y^t_{ijk} + \beta \sum_{t \in T} \sum_{i \in A} \sum_{k \in K}d^{t}_{i} m_{ik} x^t_{ik} + \gamma \sum_{t \in T} \sum_{k \in K}{c_k^t z_k^t}
    \label{eq:complete-obj}
\end{align}
\vspace*{-6mm}
\begin{align}
    \text{s.t.\quad}
    \label{eq:complete-c1}
    & v_k^t = \sum_{i \in A}{d^t_i x_{ik}^t}        &   & \forall k \in K, \forall t \in T                                  \\
    \label{eq:complete-c2}
    & \sum_{k \in K}{x_{ik}^t} = 1                  &   & \forall i \in A, \forall t \in T                                  \\
    \label{eq:complete-c3}
    & x_{ik}^t = \sum_{l \in K}{y_{ilk}^t}          &   & \forall i \in A, \forall k \in K, \forall t \in T \setminus \{1\} \\
    \label{eq:complete-c4}
    & x_{ik}^t = \sum_{l \in K}{y_{ikl}^{t+1}}      &   & \forall i \in A, \forall k \in K, \forall t \in T \setminus \{T\} \\
    \label{eq:complete-c5}
    & z_k^1 + e_k^1 + p_k \geq v_k^1 + g_k^1        &   & \forall k \in K                                                   \\
    \label{eq:complete-c6}
    & z_k^t + e_k^t + g_k^{t-1} \geq v_k^t + g_k^t  &   & \forall k \in K, \forall t \in T \setminus \{1\}                  \\
    \label{eq:complete-c7}
    & g_k^t \leq G_k                                &   & \forall k \in K, \forall t \in T                                  \\
    \label{eq:complete-c8}
    & v_k^t \leq C_k                                &   & \forall k \in K, \forall t \in T                                  \\
    \label{eq:complete-c9}
    & x_{ik}^t \in \{0,1\}                          &   & \forall i \in A, \forall k \in K, \forall t \in T                 \\
    \label{eq:complete-c10}
    & y_{ik'k''}^t \in \{0,1\}                      &   & \forall i \in A, \forall k', k'' \in K, \forall t \in T           \\
    \label{eq:complete-c11}
    & x, y, g, v, z \geq 0                          &   &
\end{align}


\dots


%
%   MODELLO MIGRAZIONI A 0
%
\section{Modello con migrazioni a costo 0}
\label{sec:modello-migrazioni-0}

\dots

\begin{align}
    \min\quad       & \beta \sum_{i \in A} \sum_{k \in K}{d_{i} m_{ik} x_{ik}} + \gamma \sum_{k \in K}{c_k z_k}
    \label{eq:migr0-obj}
\end{align}
\vspace*{-10mm}
\begin{align}
    \text{s.t.\quad}
    \label{eq:migr0-c1}
    & \sum_{i \in A}{d_i x_{ik}} = v_k              &   & \forall k \in K                  \\
    \label{eq:migr0-c2}
    & \sum_{k \in K}{x_{ik}} = 1                    &   & \forall i \in A                  \\
    \label{eq:migr0-c3}
    & z_k + e_k + p_k = v_k + g_k                   &   & \forall k \in K                  \\
    \label{eq:migr0-c4}
    & v_k \leq C_k                                  &   & \forall k \in K                  \\
    \label{eq:migr0-c5}
    & x_{ik} \in \{0,1\}                            &   & \forall i \in A, \forall k \in K \\
    \label{eq:migr0-c6}
    & a, x, p, v, z \geq 0                          &   &
\end{align}


\dots


%
%   MODELLO MIGRAZIONI A INF
%
\section{Modello con migrazioni a costo infinito}
\label{sec:modello-migrazioni-inf}

\dots

\begin{align}
    \min\quad       & \beta \sum_{t \in T} \sum_{i \in A} \sum_{k \in K}{d^t_i m_{ik} x_{ik}} + \gamma \sum_{t \in T} \sum_{k \in K}{c_k^t z_k^t}
    \label{eq:migrinf-obj}
\end{align}
\vspace*{-6mm}
\begin{align}
    \text{s.t.\quad}
    \label{eq:migrinf-c1}
    & v_k^t = \sum_{i \in A}{d^t_i x_{ik}}          &   & \forall k \in K, \forall t \in T                 \\
    \label{eq:migrinf-c2}
    & \sum_{k \in K}{x_{ik}} = 1                    &   & \forall i \in A                                  \\
    \label{eq:migrinf-c3}
    & z_k^1 + e_k^1 + p_k \geq v_k^1 + g_k^1        &   & \forall k \in K                                  \\
    \label{eq:migrinf-c4}
    & z_k^t + e_k^t + g_k^{t-1} \geq v_k^t + g_k^t  &   & \forall k \in K, \forall t \in T \setminus \{1\} \\
    \label{eq:migrinf-c5}
    & g_k^t \leq G_k                                &   & \forall k \in K, \forall t \in T                 \\
    \label{eq:migrinf-c6}
    & v_k^t \leq C_k                                &   & \forall k \in K, \forall t \in T                 \\
    \label{eq:migrinf-c7}
    & x_{ik} \in \{0,1\}                            &   & \forall i \in A, \forall k \in K                 \\
    \label{eq:migrinf-c8}
    & x, g, v, z \geq 0                             &   &
\end{align}


\dots

\chapter{Algoritmi euristici}
\label{cap:algoritmi-euristici}

Nel corso del capitolo vengono presentati i due algoritmi risolutivi euristici proposti in questo lavoro.


%
%   EURISTICA 1: SEMPLIFICAZIONE MODELLO
%
\section{Euristica 1: Semplificazione del modello}
\label{sec:semplificazione-modello}

Il primo algoritmo euristico tenta di abbassare la complessità del problema andando a semplificare il modello di ottimizzazione utilizzato. In particolare, l'idea è quella di effettuare le orchestrazioni solo in determinati time-slot prestabiliti, così da poter suddividere l'istanza considerata in sottoistanze, all'interno delle quali ogni AP rimane associato alla stessa facility per tutto l'arco temporale, e poter risolvere ciascuna indipendentemente in modo statico usando solutori generici. I risultati ottenuti verranno poi ri-aggregati per formare quello dell'istanza iniziale. Nel risultato ottenuto, le orchestrazioni vengono quindi effettuate solo in fascie consecutive appartenenti a sottoistanze diverse, ed essendo una conseguenza dell'aggregazione dei risultati parziali, non considerano i costi di migrazione. Ciò significa che verrà premiata la latenza e la gestione energetica, ma andando a pagare un costo di gestione della rete maggiore a quello previsto dalla soluzione ottima. Idealmente, per avere un buon trade-off di ottimizzazione, l'istanza inziale non deve essere suddivisa in troppe parti, e di conseguenza è ragionevole voler pagare poche volte alti costi di migrazione, pur di avere nel lungo periodo una migliore latenza e gestione dell'energia.


\subsection{Descrizione dell'algoritmo}
\label{subsec:algoritmo-sempl-modello}

\begin{algorithm}
    \caption{Pseudocodice euristica 1}
    \label{alg:euristica-semplificazione-modello}
    \begin{algorithmic}[1]
        \Require l'istanza da risolvere, composta dagli insiemi $T$, $A$, $K$ e i dati $C$, $G$, $d$, $l$, $m$, $e$, $c$
        \Ensure il piano degli assegnamenti risultante, quindi il valore delle variabili $x$
        \State Siano $x, g, v, z, s$ le variabili del risultato globale
        \State $t \gets -1$
        \State $\operatorname{costo\_migrazione} \gets 0$
        \State $S \gets \operatorname{split}(T, A, K, C, G, d, l, m, e, c)$
        \For{$\textbf{all} ~ j \in \{0, \dots, |S|-1\}$}
            \State \{Siano $j$ considerati in ordine crescente\}
            \State $(\bar{T}$, $\bar{A}$, $\bar{K}, \bar{C}, \bar{G}, \bar{d}, \bar{l}, \bar{m}, \bar{e}, \bar{c}) \gets S[j]$
            \If{$j = 0$}
                \State $p_k \gets 0 ~ \forall k \in K$
            \Else
                \State $p_k \gets g^t_k ~ \forall k \in K$
            \EndIf
            \State $(\hat{x}, \hat{g}, \hat{v}, \hat{z}, \hat{s}) \gets \operatorname{calcola\_assegnamenti}(\bar{T}$, $\bar{A}$, $\bar{K}, \bar{C}, \bar{G}, \bar{d}, \bar{l}, \bar{m}, \bar{e}, \bar{c})$
            \If{$j > 0$}
                \For{$\textbf{all} ~ i \in A$}
                \State $k_{\operatorname{prec}} \gets \underset{k \in K}{\arg\max} ~ x^t_{ik}$
                \State $k_{\operatorname{succ}} \gets \underset{k \in K}{\arg\max} ~ \hat{x}^0_{ik}$
                    \If{$k_{\operatorname{prec}} \neq k_{\operatorname{succ}}$}
                        \State $\operatorname{costo\_migrazione} \gets \operatorname{costo\_migrazione} ~ + ~ d^t_i \cdot l_{k_{\operatorname{prev}} k_{\operatorname{succ}}}$
                    \EndIf
                \EndFor
            \EndIf
            \State $x \gets x \cup \hat{x}$
            \State $g \gets g \cup \hat{g}$
            \State $v \gets v \cup \hat{v}$
            \State $z \gets z \cup \hat{z}$
            \State $s \gets s \cup \hat{s}$
            \State $t \gets t + |T|$
        \EndFor
    \end{algorithmic}
\end{algorithm}


L'\autoref{alg:euristica-semplificazione-modello} mostra le operazioni necessarie ad implementare l'idea euristica esposta. Come si può osservare, richiede in input un'istanza del problema e ritorna in output il piano degli assegnamenti, rappresentato delle variabili $x^t_{ik}$, che assumono valore 1 se al tempo $t \in T$ l'AP $i \in A$ è assegnato alla facility $k \in K$, altrimenti 0; dove $T$ è l'insieme dei time-slot che compongono l'istanza, mentre $A$ e $K$ gli insiemi degli AP e delle facility dell'infrastruttura MEC.

Per fornire una migliore leggibilità, i nomi dei dati e delle variabili rispecchino quelli utilizzati nei modelli di assegnamento descritti nel capitolo precedente. Da notare come i dati delle sottoistanze siano rappresentati con una barra (es. $\bar{d}$) e quelli dell'istanza globale senza (es. $d$), e allo stesso modo, le variabili delle sottoistanze con un cappello (es. $\hat{x}$) e quelle del risultato globale senza (es. $x$).

Successivamente vengono illustrate le operazioni effettuate dall'algoritmo raggruppandole in tre parti: frammentazione dell'istanza, ottimizzazione delle sottoistanze e composizione del risultato globale.

\subsubsection{Frammentazione dell'istanza}

La prima operazione effettuata dall'algoritmo consiste nel frammentare l'istanza in input utilizzando la funzione \textit{split} (riga 4), che restituisce una lista composta dalle sottoistanze ottenute. La frammentazione si effettua andando a identificare alcuni time-slot chiamati `split point' e generando un'istanza per ogni sequenza compresa tra ciascuna coppia, mentenendo il loro ordine originale. Le sottoistanze generate sono tra loro indipendenti e operano sulla stessa infrastruttura MEC di partenza.

\subsubsection{Ottimizzazione delle sottoistanze}

Successivamente si vanno a risolvere una per una tutte le sottoistanze utilizzando la funzione \textit{calcola\_assegnamenti} (riga 13), che implementa la variante con quadro energetico completo del modello di assegnamento statico, descritto nella \autoref{subsec:modello-statico-var}. Per poter riprodurre il corretto flusso di energia tra istanti temporali consecutivi ma appartenenti a sottoistanze diverse, è necessario effettuare l'esecuzione mantenendo il corretto ordine cronologico, così da poter inserire nelle batterie delle facility al primo time-slot di ogni sottoistanza, la quantità di energia che era stata immagazzinata nell'ultimo istante di quella precedente. Questo comportamento è descritto nelle righe 8:12, dove con $p_k$ viene indicata tale quantità, e all'inizio della prima sottoistanza le batterie vengono considerate vuote. Da notare come utilizzando il modello di assegnamento statico (invece che la sua variante) non sarebbe stato possibile determinare la reale quantità di energia immagazzinabile nell'ultimo istante temporale di ogni sottoistanza.

\subsubsection{Composizione del risultato globale}

Avendo risolto le sottoistanze seguendo l'ordine temporale, le variabili della soluzione globale si ottengono andando ad accumulare di volta in volta quelle parziali, come è mostrato nella righe 23:27.

Per determinare la validità delle soluzioni ottenute, devono poter essere confrontabili con quelle ottime, e quindi è necessario calcolare il valore della funzione obbiettivo del modello di assegnamento dinamico (\hyperref[eq:dinamico-obj]{equazione 3.1.1}) utilizzando le variabili della soluzione globale. Le variabili $y$, a differenze delle altre necessarie, non sono però disponibili, quindi i costi di migrazione complessivi vengono calcolati iterativamente dall'algoritmo e memorizzati nella variabile \textit{costi\_migrazione}. Il valore cercato sarà quindi dato da:
\begin{equation}
    \alpha \cdot \operatorname{costi\_migrazione} ~ + ~ \beta \sum_{t \in T} \sum_{i \in A} \sum_{k \in K}d^{t}_{i} m_{ik} x^t_{ik} ~ + ~ \gamma \sum_{t \in T} \sum_{k \in K}{c_k^t z_k^t}.
\end{equation}
Effettuando gli switch solo tra due sottoistanze adiacenti, i costi di migrazione vengono calcolati controllando per ogni AP, se la facility a cui è associato nell'ultimo time-slot della sottoistanza precedente ($k_{\text{prec}}$) sia diversa da quella del primo time-slot della sottoistanza successiva ($k_{\text{succ}}$): in questo caso il costo da pagare è dato dalla quantità di traffico da migrare per la distanza tra le due facility coinvolte, ossia $k_{\text{prec}}$ e $k_{\text{succ}}$ (righe 15:21).


%
%   EURISTICA 2: AGGREGAMENTO TIME-SLOTS
%
\section{Euristica 2: Aggregazione dei time-slot}
\label{sec:aggregamento-time-slot}

L'euristica proposta in questa sezione mira ad abbassare la complessità del problema semplificando l'istanza da risolvere. In particolare, l'idea è quella di frammentare l'arco temporale considerato e rappresentare ciascun frammento attraverso un singolo time-slot chiamato `rappresentante', dato dall'aggregazione di tutti gli istanti appartenenti alla stessa parte. In questo modo si ottiene un'istanza composta solo dai rappresentanti, ed avendo quindi un numero inferiore di istanti temporali, diventa più veloce da risolvere. In ogni fascia dell'istanza originale, vengono quindi eseguiti gli stessi assegnamenti del suo rappresentante, e di conseguenza le migrazioni verranno effettuate solo tra frammenti consecutivi. A differenza dell'euristica precedente, nell'ottimizzazione vengono considerati i costi di migrazione, anche se non saranno utilizzati i valori dei dati reali ma quelli dei rappresentanti. Da notare come in alcune situazioni venga generato un risultato globale non valido, in quanto gli assegnamenti da attuare in ogni frammento sono calcolati utilizzando esclusivamente il traffico dell'istante rappresentante, che in genere è diverso negli altri time-slot, e di conseguenza si potrebbe assegnare ad una facility più domanda di quanta è in grado di gestirne.


\subsection{Descrizione dell'algoritmo}
\label{subsec:algoritmo-aggregazione}

\begin{algorithm}
    \caption{Pseudocodice euristica 2}
    \label{alg:euristica-modello}
    \begin{algorithmic}[1]
        \Require l'istanza da risolvere, composta dai dati $\bar{C}, \bar{G}, \bar{d}, \bar{l}, \bar{m}, \bar{e}, \bar{c}$
        \Ensure il piano degli assegnamenti risultante, quindi il valore delle variabili $\hat{x}$
        \State Siano $\hat{x}, \hat{g}, \hat{v}, \hat{z}$ le variabili riguardanti il risultato completo
        \State $(L, C, G, d, l, m, e, c) \gets \operatorname{aggrega}(\bar{C}, \bar{G}, \bar{d}, \bar{l}, \bar{m}, \bar{e}, \bar{c})$ \label{algo1:l:aggrega}
        \State $x \gets \operatorname{calcola\_assegnamenti}(C, G, d, l, m, e, c)$
        \State $t \gets 0$
        \State $\operatorname{costi\_migrazione} \gets 0$
        \State \{generazione del risultato completo\}
        \For{$\textbf{all} ~ j \in \{0, \dots, |L| - 1\}$}
            \For{$\textbf{all} ~ u \in \{0, \dots, L[j] - 1\}$}
                \State $\hat{x}^t_{ik} \gets x^j_{ik} ~ \forall i \in A, \forall k \in K$
                \If{$n = 0$}
                    \State $p_k \gets 0 ~ \forall k \in K$
                \Else
                    \State $p_k \gets \hat{g}^{t-1}_k ~ \forall k \in K$
                \EndIf
                \State \{ricostruzione del traffico di energia\}
                \State $\hat{v}^t_k = \sum_{i \in A} \bar{d}^t_i x^t_{ik} ~ \forall k \in K$
                \State $\hat{g}^t_k \gets \min\{0, ~ \max\{p_k + \bar{e}^t_k- \hat{v}^t_k, ~ \bar{G}_k\}\} ~ \forall k \in K$
                \State $\hat{z}^t_k = \min\{0, ~ \hat{v}^t_k - \left(p_k + \bar{e}^t_k\right)\} ~ \forall k \in K$
                \State $t \gets t + 1$
            \EndFor
            \State \{calcolo dei costi di migrazione\}
            \If{$j > 0$}
            \For{$\textbf{all} ~ i \in A$}
                \State $k_{\operatorname{prec}} \gets \underset{k \in K}{\arg\max} ~ \hat{x}^{t-1-L[j]}_{ik}$
                \State $k_{\operatorname{succ}} \gets \underset{k \in K}{\arg\max} ~ x^{t-1}_{ik}$
                \If{$k_{\operatorname{prec}} \neq k_{\operatorname{succ}}$}
                    \State $\operatorname{costo\_migrazione} \gets \operatorname{costo\_migrazione} ~ + ~ \bar{d}^{t-1-L[j]}_i \cdot l_{k_{\operatorname{prev}} k_{\operatorname{succ}}}$
                \EndIf
            \EndFor
            \EndIf
        \EndFor
    \end{algorithmic}
\end{algorithm}


L'implementazione di questa euristica viene mostrata nell'\autoref{alg:euristica-semplificazione-istanza}. Tale algoritmo, richiede in input un'istanza del problema e ritorna il piano degli assegnamenti, dato dal valore delle variabili binarie $x^t_{ik}$, che assumono valore 1 solo se al tempo $t \in T$ l'AP $i \in A$ è associato alla facility $k \in K$, dove $T$ è l'insieme del time-slot mentre $A$ e $K$ quelli degli AP e delle facility. I dati e le variabili vengono indicati seguendo la convenzione adotatta nell'algoritmo precedente (consultare la \autoref{subsec:algoritmo-sempl-modello} per una panoramica completa). Successivamente viene illustrato il funzionamento dell'algoritmo.

\subsubsection{Semplificazione e ottimizzazione dell'istanza}

La prima operazione eseguita dall'algoritmo consiste nel semplificare l'istanza in input attraverso la funzione \textit{aggrega} (riga 4), che oltre al risultato atteso fornisce un array $L$: l'elemento $L[j]$ indica il numero di time-slot rappresentati dal $j$-esimo istante dell'istanza semplificata. Successivamente si determina il piano degli assegnamenti su tale istanza utilizzando il modello di ottimizzazione dinamico (\autoref{sec:modello-dinamico}), implementato dalla funzione \textit{calcola\_assegnamenti}. In questo caso il valore risultante delle variabili $\hat{g}$, $\hat{v}$ e $\hat{z}$ è superfluo.

\subsubsection{Composizione del risultato globale}

Il piano degli assegnamenti globale viene generato effettuando l'operazione inversa all'aggregazione sulle variabili $x$: tutti i time-slot di ogni frammento effettuano gli stessi assegnamenti determinati dal proprio rappresentante (riga 8, $t$ rappresenta lo slot-time del risultato globale considerato attualmente).

Anche in questo caso, si vuole poter confrontare le soluzioni ottenute con quelle ottime, e di conseguenza è necessario calcolare il valore della funzione obbiettivo (\hyperref[eq:dinamico-obj]{equazione 3.1.1}) utilizzando le variabili del risultato complessivo. A questo scopo, bisogna conoscere la quantità di energia acquistata da ogni facility in ogni istante, e per ottenere tali valori è necessario simulare l'intero scenario energetico (righe 14:16) considerando gli assegnamenti dettati dal piano globale. Dato un time-slot $t \in T$ e una facility $k \in K$, come prima cosa va calcolata la quantità di energia necessaria a gestire la domanda a cui è sottoposta $k$ al tempo $t$, indicata con $v^t_k$ e data dalla quantità totale di traffico rivolto agli AP che le sono assegnati:
\begin{equation}
    v^t_k = \sum_{i \in A} d^t_i x^t_{ik}.
\end{equation}
Successivamente bisgona determinare quanta energia viene accumulata, indicata con $g^t_k$ e data dalla differenza tra quella disponibile e qualla utilizzata:
\begin{equation}
    g^t_k = p_k + e^t_k- v^t_k
\end{equation}
da notare come questo valore non possa essere negativo e non debba poter superare la capienza della batteria ($G_k$). Con $p_k$ viene indicata la quantità di energia disponibile all'interno delle batterie all'inizio dell'istante temporale. Infine, si arriva a calcolare l'energia acquistata, data dalla differenza tra la quantità utilizzata e la quantità disponibile:
\begin{equation}
    z^t_k = v^t_k - \left(p_k + e^t_k\right)
\end{equation}
anche questo valore deve essere posto a non negativo, in modo da poter gestire correttamente i casi in cui la disponibilità sia maggiore dell'utilizzo. Da notare come nello scenario generato, l'energia acquistata debba essere direttamente utilizzata nell'istante attuale, senza poterla accumulare, e di conseguenza il risultato ottenuto potrebbe non rappresentare quello reale nei casi in cui il costo dell'energia nelle facility non sia costante nel tempo. A questo punto è possible calcolare il valore della funzione obbiettivo:
\begin{equation}
    \alpha \cdot \operatorname{costi\_migrazione} ~ + ~ \beta \sum_{t \in T}\sum_{i \in A}\sum_{k \in K} d^t_i m_{ik} x^t_{ik} ~ + ~ \gamma \sum_{t \in T} \sum_{k \in K} c^t_k z^t_k
\end{equation}
dove \textit{costi\_migrazione} sono i costi di migrazione complessivi, ottenuti nello stesso modo dell'euristica precedente.\\


%
%       APPENDICI
%

\begin{appendices}
    \chapter{Modelli di ottimizzazione aggiuntivi}
\label{cap:modelli-di-ottimizzazione-framm}

\section{Frammentazione di array in blocchi della stessa somma}
\label{sec:framm-equals}

Dato un array, l'obbiettivo di questo modello è quello di frammentarlo in un numero predefinito di blocchi aventi approssimativamente la stessa somma.


\paragraph{Dati.} Sia $I$ l'insieme dei blocchi e $J$ quello degli indici dell'array. Il numero di blocchi che si intendono ottenere è indicato con $n$, e il valore del $j$-esimo elemento viene rappresentato con $v_j$ ($j \in J$). Con $m$ si denota invece la somma che avrebbe ciascun blocco nel caso in cui sia possibile attuare una frammentazione perfetta, vale a dire $m = \frac{1}{|I|}\sum_{j \in J}v_j$.

\paragraph{Variabili.} Siano $x^i_j$ variabili binarie che assumono valore 1 se il $j$-esimo elemento dell'array fa parte dell'$i$-esimo blocco, 0 altrimenti. Le variabili $s^i_j$ sono anch'esse binarie e assumono valore 1 se il primo elemento dl'$i$-esimo blocco è il $j$-esimo elemento dell'array, altrimenti 0. Le variabili $d^i$ definiscono la distanza tra la somma del blocco $i$ e il valore ottimo $m$. Considerando $i \in I$ e $j \in J$.

\paragraph{Modello.} Di seguito la rappresentazione matematica.

\input{modelli/modello-framm-somma.tex}

\paragraph{Obbiettivo.} L'obbiettivo (\hyperref[eq:somma-obj]{equazione A.1.1}) è quello di ottenere somme dei blocchi il più possibile vicine tra loro, andando a minimizzare la somma delle distanze che li separa dal valore ottimo $m$.

\paragraph{Vincoli.} I vincoli \ref{eq:somma-c1} impongono ad ogni elemento dell'array di far parte di un singolo blocco, mentre i \ref{eq:somma-c2} fanno in modo che ogni blocco contenga almeno un elemento. I vincoli \ref{eq:somma-c3}, \ref{eq:somma-c4}, \ref{eq:somma-c5}, \ref{eq:somma-c6}, \ref{eq:somma-c7} impongono ad ogni blocco di avere un singolo indice di partenza e fanno in modo che i suoi elementi siano consecutivi. Da notare come la presenza dei vincoli \ref{eq:somma-c2} sia dovuta al fatto che i \ref{eq:somma-c6} non garantiscono che ogni blocco abbia un singolo indice iniziale, ma che se esiste allora è unico. I vincoli \ref{eq:somma-c8} e \ref{eq:somma-c9} sono l'implenentazione in programmazione lineare del concetto di valore assoluto, e definiscono il valore delle variabili $d$:
\begin{align}
    &d^i = \left| \sum_{j \in J}x^i_jv_j - m \right| && \forall i \in I
\end{align}

\subsection{Algoritmo euristico}

Quando la dimensione dell'array o quelle del numero di blocchi da ricavare aumenta, il tempo necessario a determinare la soluzione ottima cresce velocemente, e di conseguenza per semplificare la complessità del modello viene introdotto un algoritmo euristico. Il problema viene semplificato scomponendo l'array in $m$ parti composte da $n/m$ elementi: ciascuna viene risolta indipendentemente, imponendo di generare un numero di blocchi che corrisponde all'incisione percentuale della somma dei suoi elementi rispetto alla somma dell'intero array. Nelle ottimizzazioni effettuate, la funzione obbiettivo deve minimizzare la distanza della somma dei blocchi rispetto allo stesso valore, quindi in ciascuna, $m$ deve essere calcolato consideranto l'array completo e non la sottoparte da risolvere. Questo algoritmo non determina la soluzione ottima, ma fornisce una buona approssimazione che peggiora con l'aumentare delle scomposizioni attuate.


\section{Frammentazione di array in blocchi omogenei}
\label{sec:framm-omogenei}

Dato un array, lo scopo di questo modello di ottimizzazione è quello di frammentarlo in un numero predefinito di blocchi in modo da minimizzare la differenza dei suoi elementi all'interno di ciascun blocco. Il problema descritto è una versione del p-median problem on a line, risolvibile in tempo polinomiale tramite programmazione dinamica \cite{HASSIN1991395}.

\paragraph{Dati.} Sia $I$ l'insieme degli elementi dell'array, $d_{ij}$ una misura della differenza tra gli elementi $i \in I$ e $j \in I$, e $p$ un parametro che definisce quanti blocchi si intende generare.

\paragraph{Variabili.} Le variabili $y_j$ valgono $1$ se l'elemento $j \in I$ è scelto come rappresentante, $0$ altrimenti. Le variabili $x_{ij}$ valgono $1$ se l'elemento $i \in I$ è inserito in un blocco il cui rappresentante è $j \in I$, $0$ altrimenti.

\paragraph{Modello.} Di seguito la rappresentazione matematica del modello.

\begin{align}
    \min\quad       & \sum_{i \in I} \sum_{j \in I} d_{ij} x_{ij}
    \label{eq:pmedian-obj}
\end{align}
\vspace*{-6mm}
\begin{align}
    \text{s.t.\quad}
    \label{eq:pmedian-c1}
    & \sum_{j \in I} x_{ij} = 1 & & \forall i \in I                                                                    \\
    \label{eq:pmedian-c2}
    & x_{ij} \leq y_j           & & \forall i \in I, \forall j \in I                                                   \\
    \label{eq:pmedian-c3}
    & \sum_{j \in I} y_j = p    & &                                                                                    \\
    \label{eq:pmedian-c4}
    & x_{ij} \leq x_{kj}        & & \forall i \in I, \forall j \in I, \forall k \in I: j > i \wedge k > i \wedge k < j \\
    \label{eq:pmedian-c5}
    & x_{ij} \leq x_{kj}        & & \forall i \in I, \forall j \in I, \forall k \in I: j < i \wedge k < i \wedge k > j \\
    \label{eq:pmedian-c6}
    & x_{ij} \in \{0,1\}        & & \forall i \in I, \forall j \in I                                                   \\
    \label{eq:pmedian-c7}
    & y_j \in \{0,1\}           & & \forall j \in I
\end{align}


\paragraph{Obbiettivo.} La funzione obiettivo \eqref{eq:pmedian-obj} minimizza la somma delle differenze tra un elemento e il rappresentante del blocco in cui è inserito.

\paragraph{Vincoli.} I vincoli \eqref{eq:pmedian-c1} assicurano che ogni elemento sia inserito in un blocco. I vincoli \eqref{eq:pmedian-c2} assicurano che un elemento possa essere inserito in un blocco il cui rappresentante è $j \in I$ solo se $j$ è identificato come un rappresentante. I vincoli \eqref{eq:pmedian-c4} assicurano che se ci sono tre elementi $i, k, j \in I \times I \times I$ che compaiono uno dopo l'altro in ordine, e $i$ è assegnato a $j$, allora anche $k$ è assegnato a $j$. Simmetricamente, i vincoli \eqref{eq:pmedian-c5} assicurano che se ci sono tre elementi $j, k, i \in I \times I \times I$ che compaiono uno dopo l'altro in ordine, e $i$ è assegnato a $j$, allora anche $k$ è assegnato a $j$.

\end{appendices}

%
%       BIBLIOGRAFIA
%

\bibliographystyle{unsrt}
\bibliography{bibliografia}
\addcontentsline{toc}{chapter}{Bibliografia}


% Pagina dichiusura del LIM
% \closingpage

\end{document}
