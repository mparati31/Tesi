\chapter{Modelli di ottimizzazione}
\label{cap:modelli-ottimizzazione}

In questo capitolo vengono introdotti i modelli di ottimizzazione utilizzati nel corso dell'elaborato. Nel primo sottocapitolo si illustra il modello di assegnamento dinamico rappresentante il problema descritto, mentre nei due successivi quelli utilizzati dagli algoritmi euristici: il modello di assegnamento statico ed il modello che effettua l'assegnamento utilizzando istanze composte da un solo time-slot.


%
%   MODELLO COMPLETO
%
\section{Modello di assegnamento dinamico}
\label{sec:modello-completo}

Il `modello di assegnamento dinamico' è un modello di programmazione lineare misto-intera (MILP) che, considerata un'infrastruttura MEC, è in grado di definire dinamicamente gli assegnamenti AP-facility all'interno di un arco temporale in cui sono noti i livelli di traffico rivolti agli AP e l'energia prodotta dalle facility, per ogni time-slot. Con `dinamicamente' si intende la possibilità di assegnare lo stesso AP a facility differenti nel corso dei time-slot, permettento di bilanciare nel tempo la domanda complessiva rivolta alle facility. Questo migliora la latenza complessiva e la gestione energetica attuata, ma provoca la necessità di dover gestire i vari switch e pagare i costi di migrazione. L'output del modello è un piano degli assegnamenti: viene indicato in ogni time-slot a quale facility della rete deve indirizzare il traffico ogni AP. Di conseguenza è possibile estrarre anche il piano delle orchestrazioni, che corrisponde ad un valore booleano per ogni time-slot, per ogni AP e per ogni coppia di facility, che assume come valore `vero' se in quell'instante temporale l'AP effettua uno switch tra le facility considerate, altrimenti `falso'.

\paragraph*{Dati.}

Data un'infrastruttura MEC, siano $A$ l'insieme degli AP e $K$ l'insieme delle facility che la compongono. Si supponga inoltre di avere un orizzonte temporale discretizzato in un insieme di $T$ time-slot. Vengono indicati, per ogni AP $i \in A$, con $d^t_i$ la quantità di traffico a cui $i$ è sottoposto al time-slot $t \in T$ e con $m_{i,k}$ la distanza fisica tra $i$ e la facility $k \in K$. La distanza di rete tra due facility $j, k \in K \times K$ è invece denotata con $l_{jk}$: tale valore è direttamente proporzionale alla loro distanza fisica ed alla latenza di rete, includendo anche il tempo necessario a processare i pacchetti all'interno dei nodi intermedi. Data una facility $k \in K$, siano: $C_k$ la quantità di domanda che riesce a gestire contemporaneamente, $G_k$ la capienza complessiva delle sue batterie, $p_k$ l'energia inizialmente presente all'interno delle batterie, $e^t_k$ l'energia che produce nel time-slot $t \in T$, e $c^t_k$ il prezzo a cui può acquistare l'energia nel time-slot $t \in T$.

\paragraph*{Variabili.}

Siano $x^t_{ik}$ variabili binarie che assumono valore 1 nel caso in cui al time-slot $t \in T$ l'AP $i \in A$ è assegnato alla facility $k \in K$, altrimenti 0. Le variabili $y^t_{ijk}$ sono anch'esse binarie e valgono 1 se l'AP $i \in A$ viene associato alla facility $j \in K$ al tempo $t - 1$ ed alla facility $k \in K$ al tempo $t$, altrimenti 0. Da notare come queste siano le variabili che indicano le orchestrazioni da effettuare. Per ogni facility $k \in K$ e time-slot $t \in T$, siano: $g^t_k$ l'energia residua della facility $k$ al tempo $t$ (quindi quella che viene immagazzinata all'interno delle batterie e disponibile nel time-slot $t+1$), $v^t_k$ l'energia utilizzata dalla facility $k$ al tempo $t$ e $z^t_k$ la quantità di energia acquistata dalla facility $k$ al tempo $t$.

\paragraph*{Modello.}

Di seguito la rappresentazione matematica del modello.

\begin{align}
    \min\quad       & \alpha \sum_{t \in T} \sum_{i \in A} \sum_{\substack{(j,k) \in \\ K \times K}}d^{t}_{i} l_{jk} y^t_{ijk} + \beta \sum_{t \in T} \sum_{i \in A} \sum_{k \in K}d^{t}_{i} m_{ik} x^t_{ik} + \gamma \sum_{t \in T} \sum_{k \in K}{c_k^t z_k^t}
    \label{eq:complete-obj}
\end{align}
\vspace*{-6mm}
\begin{align}
    \text{s.t.\quad}
    \label{eq:complete-c1}
    & v_k^t = \sum_{i \in A}{d^t_i x_{ik}^t}        &   & \forall k \in K, \forall t \in T                                  \\
    \label{eq:complete-c2}
    & \sum_{k \in K}{x_{ik}^t} = 1                  &   & \forall i \in A, \forall t \in T                                  \\
    \label{eq:complete-c3}
    & x_{ik}^t = \sum_{l \in K}{y_{ilk}^t}          &   & \forall i \in A, \forall k \in K, \forall t \in T \setminus \{1\} \\
    \label{eq:complete-c4}
    & x_{ik}^t = \sum_{l \in K}{y_{ikl}^{t+1}}      &   & \forall i \in A, \forall k \in K, \forall t \in T \setminus \{T\} \\
    \label{eq:complete-c5}
    & z_k^1 + e_k^1 + p_k \geq v_k^1 + g_k^1        &   & \forall k \in K                                                   \\
    \label{eq:complete-c6}
    & z_k^t + e_k^t + g_k^{t-1} \geq v_k^t + g_k^t  &   & \forall k \in K, \forall t \in T \setminus \{1\}                  \\
    \label{eq:complete-c7}
    & g_k^t \leq G_k                                &   & \forall k \in K, \forall t \in T                                  \\
    \label{eq:complete-c8}
    & v_k^t \leq C_k                                &   & \forall k \in K, \forall t \in T                                  \\
    \label{eq:complete-c9}
    & x_{ik}^t \in \{0,1\}                          &   & \forall i \in A, \forall k \in K, \forall t \in T                 \\
    \label{eq:complete-c10}
    & y_{ik'k''}^t \in \{0,1\}                      &   & \forall i \in A, \forall k', k'' \in K, \forall t \in T           \\
    \label{eq:complete-c11}
    & x, y, g, v, z \geq 0                          &   &
\end{align}


\paragraph*{Obbiettivo.}

La funzione obbiettivo (\ref{eq:complete-obj}) ha lo scopo di ottenere la soluzione che possiede il miglior bilanciamento tra avere una buona latenza per l'utente finale ed ottenere costi di migrazione e di acquisto dell'energia contenuti. In particolare, si vuole minimizzare la somma tra (nell'ordine) i costi complessivi di migrazione, la latenza complessiva e il costo globale di acquisto dell'energia, dove i parametri $\alpha$, $\beta$ e $\gamma$ indicano quanto peso dare a ciascun componente.

\paragraph*{Vincoli.}

I vincoli \ref{eq:complete-c1} determinano il valore delle variabili $v^t_k$: la domanda a cui è sottoposta la facility $k \in K$ al tempo $t \in T$ è data dalla somma del traffico proveniente da tutti gli AP che le sono assegnati in quel time-slot. I vincoli \ref{eq:complete-c2} impongono ogni AP ad essere assegnato esattamente ad una facility per time-slot. I vincoli \ref{eq:complete-c3} e \ref{eq:complete-c4} collegano le variabili $x$ ed $y$ e servono a gestire il flusso degli switch: quando $x^t_{ik}$ vale 0 significa che l'AP $i \in A$ non è associato alla facility $k \in K$ e di conseguenza non permettono di effettuare nessuna orchestrazione, mentre nel caso contrario impongono di effettuarne esattamente una oppure mantenere l'AP $i$ associato alla facility $k$. I vincoli \ref{eq:complete-c5} e \ref{eq:complete-c6} gestiscono il flusso dell'energia: impongono che per ogni time-slot e per ogni facility la quantità di energia disponibile (data dalla somma tra quella prodotta, quella presente nelle batterie e quella acquistata) sia non minore della somma tra la quantità utilizzata e quella che viene immagazzinata nelle batterie. I vincoli \ref{eq:complete-c7} e \ref{eq:complete-c8} limitano superiormente i valori che possono assumere rispettivamente le variabili $g$ e $v$, facendo in modo che non venga sforata la capienza delle batterie e la capacità delle facility.


%
%   MODELLO MIGRAZIONI A INF
%
\section{Modello di assegnamento statico}
\label{sec:modello-migrazioni-inf}

Il `modello di assegnamento statico' è un modello di tipo MILP che ha lo stesso scopo del modello di assegnamento dinamico (\ref{sec:modello-completo}) ma con la differenza che non permette di effettuare switch. Ogni AP sarà quindi assegnato alla stessa facility per tutto l'arco temporale preso in considerazione, e non attuando migrazioni il costo di migrazione complessivo sarà 0. In generale, utilizzare un approccio di assegnamento dinamico fornisce minore latenza e migliore efficienza energetica complessive rispetto ad uno statico, in quanto è possibile adattare gli assegnamenti alla variazione del traffico richiesto dai vari AP ed alla quantità di energia prodotta dalle facility. La formulazione del problema statico è però più leggera rispetto alla controparte dinamica, infatti necessita di un numero inferiore di variabili e vincoli, con conseguente diminuzione della complessità combinatoria. L'output del modello in questo caso sarà una lista di assegnamenti validi per tutti i time-slot.

\paragraph*{Dati.}

I dati su cui opera sono gli stessi del modello dinamico, si veda la \autoref{sec:modello-completo} per una panoramica completa.

\paragraph*{Variabili.}

Siano $x_{ik}$ variabili binarie che assumono valore 1 se l'AP $i \in A$ viene assegnato alla facility $k \in K$, 0 altrimenti. Per ogni facility $k \in K$ e per ogni time-slot $t \in T$ siano inoltre $g^t_k$ l'energia residua della facility $k$ al tempo $t$, $v^t_k$ l'energia utilizzata dalla facility $k$ al tempo $t$ e $z^t_k$ la quantità di energia acquistata dalla facility $k$ al tempo $t$. Come si può notare, il set di variabili viene notevolmente ridotto rispetto al modello dinamico, in quanto imponendo assegnamenti statici non è più necessario gestire il flusso degli switch e memorizzare gli assegnamenti per ogni time-slot. Le variabili $y$ diventano quindi superflue, e per determinare gli assegnamenti è sufficiente una variabile $x$ per ogni coppia AP-facility.

\paragraph*{Modello.}

Di seguito la rappresentazione matematica del modello.

\begin{align}
    \min\quad       & \beta \sum_{t \in T} \sum_{i \in A} \sum_{k \in K}{d^t_i m_{ik} x_{ik}} + \gamma \sum_{t \in T} \sum_{k \in K}{c_k^t z_k^t}
    \label{eq:migrinf-obj}
\end{align}
\vspace*{-10mm}
\begin{align}
    \text{s.t.\quad}
    \label{eq:migrinf-c1}
    & \sum_{i \in A}{d^t_i x_{ik}} = v_k^t          &   & \forall k \in K, \forall t \in T                 \\
    \label{eq:migrinf-c2}
    & \sum_{k \in K}{x_{ik}} = 1                    &   & \forall i \in A                                  \\
    \label{eq:migrinf-c3}
    & z_k^1 + e_k^1 + p_k \geq v_k^1 + g_k^1        &   & \forall k \in K                                  \\
    \label{eq:migrinf-c4}
    & z_k^t + e_k^t + g_k^{t-1} \geq v_k^t + g_k^t  &   & \forall k \in K, \forall t \in T \setminus \{1\} \\
    \label{eq:migrinf-c5}
    & g_k^t \leq G_k                                &   & \forall k \in K, \forall t \in T                 \\
    \label{eq:migrinf-c6}
    & v_k^t \leq C_k                                &   & \forall k \in K, \forall t \in T                 \\
    \label{eq:migrinf-c7}
    & x_{ik} \in \{0,1\}                            &   & \forall i \in A, \forall k \in K                 \\
    \label{eq:migrinf-c8}
    & x, g, v, z \geq 0                             &   &
\end{align}


\paragraph*{Obbiettivo.}

La funzione obbiettivo (\ref{eq:migrinf-obj}) ha lo scopo di determinare la soluzione che rappresenta il miglior compromesso tra ottenere una bassa latenza complessiva e un prezzo totale di acquisto dell'energia contenuto. I parametri $\beta$ e $\gamma$ corrispondono ai pesi da assegnare a ciascuna componente. Da notare come l'unica differenza con la funzione del modello normale (\ref{eq:complete-obj}) sia l'assenza della prima parte, riguardante i costi di migrazione.

\paragraph*{Vincoli.}

I vincoli \ref{eq:migrinf-c1} determinano il valore delle variabili $v^t_k$: l'energia utilizzata dalla facility $k \in K$ al tempo $t \in T$ è data dalla somma del traffico proveniente da ogni AP $i \in A$ che le è assegnato; mentre i vincoli \ref{eq:migrinf-c2} impongono ad ogni AP di essere associato esattamente ad una facility. I vincoli \ref{eq:migrinf-c3} e \ref{eq:migrinf-c4} riguardano la gestione ed il flusso dell'energia e sono equivalenti ai \ref{eq:complete-c5} e \ref{eq:complete-c6} del modello dinamico, mentre quelli rappresentati dalle disequazioni \ref{eq:migrinf-c5} e \ref{eq:migrinf-c6} impongono di non eccedere rispettivamente la capacità delle batterie e la capienza delle facility, e sono l'equivalente dei \ref{eq:migrinf-c7} e \ref{eq:migrinf-c8}.


\subsection{Variante con quadro energetico completo}
\label{subsec:modello-statico-var}

Il modello presentato non garantisce di tracciare il reale flusso di energia che avviene all'interno delle facility, dato che per come sono posti i vincoli \ref{eq:migrinf-c3} e \ref{eq:migrinf-c4} ci potrebbero essere delle quantità di energia residua che non vengono accumulate in quanto non utilizzate nei time-slot successivi. Nel determinare la soluzione ottima questo non causa problemi, ma quando si necessita di un quadro realistico della gestione energetica è necessario apportare alcune modifiche. In particolare, si deve immagazzinare ogni volta tutta l'energia possibile, quindi i vincoli \ref{eq:migrinf-c3} e \ref{eq:migrinf-c4} vengono rispettivamente riformulati come:
\begin{align}
    \label{eq:statico-c3var}
    & z_k^1 + e_k^1 + p_k = v_k^1 + g_k^1 + s_k^1       &   & \forall k \in K                                  \\
    \label{eq:statico-c4var}
    & z_k^t + e_k^t + g_k^{t-1} = v_k^t + g_k^t + s_k^t &   & \forall k \in K, \forall t \in T \setminus \{1\}
\end{align}
\noindent
dove $s^t_k$ sono variabili non negative che indicano la quantità di energia non accumulabile nella facility $k \in K$ al tempo $t \in T$ a causa della capienza limitata delle batterie. La quantità di energia disponibile (data dalla somma tra quella acquistata, quella prodotta e quella presente nelle batterie) è posta quindi uguale alla somma tra la quantità utilizzata, la quantità accumulata e la quantità non accumulabile. A questo punto nella funzione obbiettivo (\ref{eq:migrinf-obj}) è necessario minimizzare anche la somma complessiva dell'energia non accumulabile ($\sum_{t \in T}\sum_{k \in K} s^t_k$), trasformandola in:
\begin{align}
    \label{eq:migrinf-objvar}
    \min\quad       & \beta \sum_{t \in T} \sum_{i \in A} \sum_{k \in K}{d^t_i m_{ik} x_{ik}} + \gamma \sum_{t \in T} \sum_{k \in K}{c_k^t z_k^t} + \sum_{t \in T} \sum_{k \in K}{s_k^t}
\end{align}
\noindent
in modo da ottenere nelle variabili $g$ il valore più alto possibile, e quindi immagazzinare ogni volta tutta la quantità possibile.\\
Di seguito la rappresentazione matematica completa:

\begin{align}
    \tag{\ref*{eq:statico-objvar}}
    \min\quad       & \beta \sum_{t \in T} \sum_{i \in A} \sum_{k \in K}{d^t_i m_{ik} x_{ik}} + \gamma \sum_{t \in T} \sum_{k \in K}{c_k^t z_k^t} + \sum_{t \in T} \sum_{k \in K}{s_k^t}
\end{align}
\vspace*{-6mm}
\begin{align}
    \text{s.t.\quad}
    \tag{\ref*{eq:statico-c1}}
    & v_k^t = \sum_{i \in A}{d^t_i x_{ik}}              &   & \forall k \in K, \forall t \in T                 \\
    \tag{\ref*{eq:statico-c2}}
    & \sum_{k \in K}{x_{ik}} = 1                        &   & \forall i \in A                                  \\
    \tag{\ref*{eq:statico-c3var}}
    & z_k^1 + e_k^1 + p_k = v_k^1 + g_k^1 + s_k^1       &   & \forall k \in K                                  \\
    \tag{\ref*{eq:statico-c4var}}
    & z_k^t + e_k^t + g_k^{t-1} = v_k^t + g_k^t + s_k^t &   & \forall k \in K, \forall t \in T \setminus \{1\} \\
    \tag{\ref*{eq:statico-c5}}
    & g_k^t \leq G_k                                    &   & \forall k \in K, \forall t \in T                 \\
    \tag{\ref*{eq:statico-c6}}
    & v_k^t \leq C_k                                    &   & \forall k \in K, \forall t \in T                 \\
    \tag{\ref*{eq:statico-c7}}
    & x_{ik} \in \{0,1\}                                &   & \forall i \in A, \forall k \in K                 \\
    \label{eq:statico-c8}
    & x, g, v, z, s \geq 0                              &   &
\end{align}

Da notare come l'aggiunta delle variabili $s$ renda questo modello leggermente più complesso rispetto a quello di riferimento, e che nel risolvere le stesse istanze si otterranno risultati della stessa qualità.


%
%   MODELLO SINGOLO SLOT TIME
%
\section{Modello di assegnamento singolo time-slot}
\label{sec:modello-migrazioni-0}

Il `modello di assegnamento singolo time-slot' è un modello di tipo MILP ottimizzato per definire gli assegnamenti all'interno di istanze composte da un singolo time-slot, e fornisce quindi in output la facility a cui deve indirizzare il traffico ogni AP nel time-slot considerato.

\paragraph*{Dati.}

Data un'infrastruttura MEC, siano $A$ l'insieme degli AP e $K$ l'insieme delle facility che la compongono. Per ogni AP $i \in A$, vengono indicati con $d_i$ la quantità di traffico a cui $i$ è sottoposto e con $m_{ik}$ la distanza fisica tra $i$ e la facility $k \in K$. La distanza di rete tra due facility $j, k \in K \times K$ è invece denotata con $l_{jk}$: tale valore è direttamente proporzionale alla loro distanza fisica ed alla latenza di rete, incluso il tempo di processamento dei pacchetti nei nodi intermedi. Data una facility $k \in K$, siano: $C_k$ la quantità di domanda che riesce a gestire contemporaneamente, $p_k$ l'energia inizialmente presente all'interno delle sue batterie, $e_k$ l'energia che produce, e $c_k$ il prezzo a cui può acquistare l'energia.

\paragraph*{Variabili.}

Gli assegnamenti sono rappresentati da variabili binarie $x_{ik}$, che assumono come valore 1 se l'AP $i \in A$ viene assegnagno alla facility $k \in K$, altrimenti 0. Inoltre, data una facility $k \in K$, siano: $z_k$ l'energia che acquista, $v_k$ l'energia che utilizza e $s_k$ l'energia residua al termine del time-slot.

\paragraph*{Modello.}

\begin{align}
    \min\quad       & \beta \sum_{i \in A} \sum_{k \in K}{d_{i} m_{ik} x_{ik}} + \gamma \sum_{k \in K}{c_k z_k}
    \label{eq:migr0-obj}
\end{align}
\vspace*{-10mm}
\begin{align}
    \text{s.t.\quad}
    \label{eq:migr0-c1}
    & \sum_{i \in A}{d_i x_{ik}} = v_k              &   & \forall k \in K                  \\
    \label{eq:migr0-c2}
    & \sum_{k \in K}{x_{ik}} = 1                    &   & \forall i \in A                  \\
    \label{eq:migr0-c3}
    & z_k + e_k + p_k = v_k + g_k                   &   & \forall k \in K                  \\
    \label{eq:migr0-c4}
    & v_k \leq C_k                                  &   & \forall k \in K                  \\
    \label{eq:migr0-c5}
    & x_{ik} \in \{0,1\}                            &   & \forall i \in A, \forall k \in K \\
    \label{eq:migr0-c6}
    & a, x, p, v, z \geq 0                          &   &
\end{align}


\paragraph*{Obbiettivo.}

La funzione obbiettivo (\ref{eq:migr0-obj}) ha lo scopo di definire gli assegnamenti che rispecchiano il miglior trade-off tra l'ottenere una bassa latenza complessiva ed avere una spesa energetica contenuta. Da notare come prendendo in cosiderazione un singolo time-slot, non sia possibile effettuare switch e di conseguenza l'obbiettivo rispecchia quello presente nel modello di assegnamento statico (\ref{eq:migrinf-obj}).

\paragraph*{Vincoli.}

I vincoli \ref{eq:migr0-c1} definiscono il valore delle variabili $v_k$, mentre i \ref{eq:migr0-c2} impongono di assegnare ogni AP ad esattamente una facility. I vincoli \ref{eq:migr0-c3} gestiscono l'utilizzo dell'energia, mentre i \ref{eq:migr0-c4} definiscono un limite superiore alla quantità di traffico che ogni facility può gestire. Da notare come in questo caso, trattandosi di un singolo time-slot, non viene tenuta in considerazione la possibilità di immagazzinare l'energia nella batterie, e quindi tutta quelle le quantità in eccesso saranno indicate dalle variabili $s$.
