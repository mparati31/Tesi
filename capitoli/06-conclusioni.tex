\chapter{Conclusioni}
\label{cap:conclusioni}

In conclusione, le due euristiche proposte nelle sezioni \ref{sec:semplificazione-modello} e \ref{sec:aggregamento-time-slot}, rappresentano delle valide implementazioni per il modello di orchestrazione considerato. Negli esperimenti condotti, è stato sottolineato come una parte chiave degli algoritmi è dovuta alla loro configurazione, vale a dire il modo in cui viene suddivisa l'istanza nella prima euristica e in che modo vengono aggregate le fascie temporale nella seconda. Nello specifico, l'euristica 1 (\autoref{sec:semplificazione-modello}) esegue istanze con pochi time-slot usando un modello di ottimizzazione molto efficiente, e di consequenza riesce ad arrivare al risultato in modo veloce; l'euristica 2 (\autoref{sec:aggregamento-time-slot}) utilizza invece il modello di assegnamento dinamico, che è più complesso ma permette di ottenere le soluzioni ottime dell'istanza semplificata, e di conseguenza il tempo di calcolo è maggiore del precedente. I risultati dei test, effettuati utilizzando istanze realistiche che rappresentano l'arco di una gioranta, non riscontrano un algoritmo predominante, ma mostrano come in generale, il miglior rapporto tra il tempo di calcolo e il risultato ottento sia dato dale configurazioni che effettuano un basso numero di scomposizioni.
