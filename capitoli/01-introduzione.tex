\chapter{Introduzione}
\label{cap:introduzione}

Negli ultimi anni le reti di comunicazione hanno subito una rapida evoluzione, abbracciando la virtualizzazione dei sistemi come metodologia per ottimizzare l'efficienza delle risorse fisiche e le spese necessarie per gestire la rete, oltre che aumentare la qualità dell'esperienza per l'utente finale. Un'area di ricerca particolarmente promettente è quella riguardante il Mobile Edge Computing (MEC) (\cite{MEC}), che propone una versione distribuita del classico cloud computing, in cui i server applicativi sono istallati nell'edge della rete tramite un sistema di virtualizzazione, in modo da diminuire la distanza di rete tra gli utenti e i server, ed offrire quindi maggiore affidabilità e migliori prestazioni alle connessioni di rete, oltre che a ridurre la quantità di energia necessaria a gestire l'intera infrastruttura. Questo tipo di architettura migliora il servizio offerto all'utente riducendo drasticamente la latenza di connessione e virtualizzando le operazioni di calcolo sul nodo più vicino, permettendo di avere un basso impatto energetico sui dispositivi mobili e quindi poter eseguire applicazioni computazionalmente pesanti utilizzando device a basso consumo (\cite{7901477}). Diversi scenari applicativi traggono vantaggio dall'utilizzo dell'infrastruttura MEC, come la realtà aumentata, i veicoli a guida autonoma, l'Internet tattile (\cite{internet-tattile}), o più in generale tutti quelli in cui sia necessaria elevata potenza di computazione e bassa latenza.


%
%   OBBIETTIVI
%
\section{Obbiettivi}
\label{sec:obbiettivi}

Questo lavoro propone un modello di orchestrazione ottima che permette di gestire in modo esplicito la gestione dell'energia da parte dei server edge, considerando lo scenario in cui ogni sito ospita impianti fotovoltaici di produzione e accumulo dell'energia. Questi impianti permettono di organizzare la computazione considerando, oltre al livello di servizio rivolto agli utenti, il costo energetico pagato dovuto al calcolo, che sarà molto inferiore se si riesce ad utilizzare in maniera efficiente l'energia prodotta ed accumulata negli impianti fotovoltaici. Il modello è stato formulato secondo il formalisto della programmazione lineare, ottenendo una variante del \textit{time-dependent generalized assignment problem}, successivamente implementato attraverso due algoritmi risolutivi euristici, che verranno approfondidi, analizzati e confrontati nel corso dei capitoli.


%
%   ORGANIZZAZIONE ELABORATO
%
\section{Struttura dell'elaborato}
\label{sec:organizzazione-elaborato}

Il documento è diviso in sei capitoli:
\begin{itemize}
    \item[\textbf{1.}] \textbf{\nameref{cap:introduzione}}: introduzione al lavoro svolto;
    \item[\textbf{2.}] \textbf{\nameref{cap:modellazione-sistema}}: fornisce una panoramica riguardo l'infrastruttura Mobile Edge Computing e illustra il funzionamento del modello di orchestrazione proposto;
    \item[\textbf{3.}] \textbf{\nameref{cap:modelli-ottimizzazione}}: definisce i modelli di ottimizzazione utilizzati dagli algoritmi euristici, mostrando la loro rappresentazione matematica e descrivendo nel dettaglio le variabili, i vincoli e l'obbiettivo da raggiungere;
    \item[\textbf{4.}] \textbf{\nameref{cap:algoritmi-euristici}}: illustra i due algoritmi euristici proposti, mostrando il loro pseudocodice e descrivendo le operazioni che svolgono;
    \item[\textbf{5.}] \textbf{\nameref{cap:analisi-sperimentale}}: inizialmente presenta il setup e i dati utilizzati, successivamente descrive i test e mostra i risultati ottenuti;
    \item[\textbf{6.}] \textbf{\nameref{cap:conclusioni}}: descrive dettagliatamente le conclusioni più rilevanti ottenute nella parte precedente.
\end{itemize}
Dopo l'ultimo capitolo, è presente l'appendice \textbf{\nameref{cap:modelli-di-ottimizzazione-framm}}, in cui sono presentati due modelli di ottimizzazione che propongono due diverse metodologie per frammentare un array in blocchi.
